\documentclass[a4paper,oneside,12pt,titlepage]{scrartcl}   %Grundeinstellungen

\usepackage[ngerman]{babel}
%\usepackage[T1]{fontenc} % umlaute
\usepackage[utf8]{inputenc}
\usepackage{setspace}
\usepackage{amsfonts}
\usepackage{amsmath}
\usepackage{amssymb}
\usepackage{amsthm}
\usepackage{subfiles} 
\usepackage{caption}
\usepackage{hyperref}
\usepackage{ragged2e}
\usepackage{wrapfig}
\usepackage[amssymb]{SIunits}
\usepackage{cite}
\usepackage[usenames,dvipsnames]{xcolor}
\usepackage{listings}
\usepackage{textcomp}

\definecolor{comment}{RGB}{0,128,0} % dark green
\definecolor{string}{RGB}{255,0,0}  % red
\definecolor{keyword}{RGB}{0,0,255} % blue

\lstdefinestyle{c}{
	commentstyle=\color{comment},
	stringstyle=\color{string},
	keywordstyle=\color{keyword},
	basicstyle=\footnotesize\ttfamily,
	numbers=left,
	numberstyle=\tiny,
	numbersep=5pt,
	frame=lines,
	breaklines=true,
	prebreak=\raisebox{0ex}[0ex][0ex]{\ensuremath{\hookleftarrow}},
	showstringspaces=false,
	upquote=true,
	tabsize=2,
}
 \lstset{language=c} 


\renewcommand{\familydefault}{\sfdefault} 
\usepackage{graphicx}
\usepackage{subcaption}                       
\usepackage{float}
\usepackage[section]{placeins}
\renewcommand{\topfraction}{0.85}
\renewcommand{\textfraction}{0.1}
\renewcommand{\floatpagefraction}{0.75}

\usepackage{cite}
\usepackage[usenames,dvipsnames]{xcolor}


\usepackage{listings}

\lstset{%
	 basicstyle=\scriptsize\ttfamily,
   keywordstyle=\bfseries\ttfamily\color{NavyBlue},
   stringstyle=\color{Rhodamine}\ttfamily,
   commentstyle=\color{Green}\ttfamily,
   emph={square}, 
   emphstyle=\color{blue}\texttt,
   emph={[2]root,base},
   emphstyle={[2]\color{yac}\texttt},
   language=Python,%
   tabsize=2,%
   basicstyle=\footnotesize\ttfamily,%
   numbers=left,%
   numberfirstline,%
   breaklines=true,%
   breakatwhitespace=true,%
   linewidth=\textwidth,%
   xleftmargin=0.075\textwidth,%
   frame=tlrb,%
   captionpos=b%
}


%Header Definitionen
\usepackage{fancyhdr}
\renewcommand{\headrulewidth}{0.5pt}
\renewcommand{\footrulewidth}{0.5pt}
%Abstand zwischen Absätzen, Zeilenabstände
\voffset26pt 
\parskip6pt
%\parindent1cm  %Rückt erste Zeile eines neuen Absatzes ein
\usepackage{setspace}
\onehalfspacing

\begin{document}
\pagenumbering{arabic}

% Titeseite
\begin{titlepage}
\begin{center}
	\begin{minipage}{0.6\textwidth}
	\begin{tabular}{l}
	Technische Universität Berlin\\
	Fakulät IV (Elektrotechnik und Informatik)\\
	Institut für Energie- und Automatisierungstechnik\\
	Fachgebiet Lichttechnik\\
	\end{tabular}
	\end{minipage}
	\hfill
	\begin{minipage}{0.3\textwidth}\raggedright
	\includegraphics[scale=0.06]{tu-logo}\\
	\end{minipage}
	
	\vspace{3cm}
    \sffamily \LARGE \textbf{Entwicklung und Realisierung einer Messeinrichtung mit den Sensoren AS7261 und AS72651 von ams}\\
 	\Large Bachelorarbeit\\
    \vspace{2.5cm}
	{\renewcommand{\arraystretch}{0.7}
    	\begin{tabular}{ll}
    		Vorgelegt von: & Lennard Bödiger\\
			Studiengang:	 & Technische Informatik\\
			Matrikel-Nr.: & 363470\\
		\end{tabular}
	}

  	\vspace{5.3cm}
	\end{center}
	\begin{tabular}{ll}
		Eingereicht am: & 8. Februar 2021\\
		Betreuung: & Nils Weber\\
		Prüfer/in: & Prof. Dr.-Ing. Stephan Völker Prof.\\ 	& Dr.-Ing. Sibylle Dieckerhoff
	\end{tabular}\\
\end{titlepage}
\newpage
\begin{figure}[H]
\centering
\includegraphics[width=1\textwidth]{img/alles-ich.png}
\end{figure}

\newpage

\section*{Abstract}
\subfile{sections/Abstract}
\newpage
\tableofcontents
\newpage
\listoffigures
\newpage
\listoftables
\newpage

\section{Einleitung}
\subfile{sections/Einleitung}
\newpage
\section{Technische Grundlagen}
\subfile{sections/Technische_Grundlagen}
\newpage
\section{Hardware Komponenten}
\subfile{sections/Hardware_Komponenten}
\newpage
\section{Platine}
\subfile{sections/Platine}
\newpage
\section{Datenbank \& Webinterface}
\subfile{sections/Datenbank_und_Webinterface}
\newpage
\section{C Code}
\subfile{sections/Software}
\newpage
\section{Benutzerhandbuch}
\subfile{sections/Benutzerhandbuch}
\newpage
\section{Messungen}
\subfile{sections/Messungen}
\newpage
\section{Zusammenfassung}
Der Raspberry Pi hat sich als Steuergerät für den Messaufbau bewährt.
Der Programmcode kann schnell und einfach über SSH ausgeführt werden und auch die Fehlersuche ist sehr komfortabel.\\
Grafana lässt sich in der grafischen Oberfläche sehr einfach anpassen. Jedoch ist das einmalige Einrichten eines Dashboards für viele Sensoren umständlich und zeitaufwendig.
Die Datenkomprimierungseffizienz von InfluxDB übertrifft die Erwartungen, so dass es möglich ist, auch mit kleinen SD-Karten über mehrere Jahre Daten zu erfassen.\\
Alle Softwarekomponenten laufen im Dauertest mit einer angeschlossenen Sensorplatine seit sechs Wochen ohne Probleme.
In einem weiteren Test läuft das System mit 4 angeschlossenen Sensorboards seit zwei Wochen ebenfalls ohne Probleme.\smallskip

\noindent Zum Abgabezeitpunkt der vorliegenden Arbeit besteht das Problem, dass nur vier Sensorplatinen, d.h. 8 I2C-Slaves, gleichzeitig am Bus angeschlossen werden können.
Ab dem fünften Sensorboard kann der Raspberry Pi die I2C-Adressen aller angeschlossenen Sensoren nicht mehr finden.
Als Workaround gibt es die Möglichkeit, einen Bus-Multiplexer zu verwenden.
Dieses Gerät wird über I2C direkt an den Bus angeschlossen und hat mehrere I2C-Bus-Ports. Über einen Befehl kann dem Multiplexer mitgeteilt werden, welcher I2C-Bus-Kanal zum Raspberry Pi durchgeschaltet wird.
Eine Betaversion des Workarounds ist bereits implementiert und auf Github\footnote{\url{https://github.com/LennardBoediger/Bachelorarbeit}} zu finden, wird aber in dieser Arbeit nicht behandelt.
\smallskip

\noindent Abgesehen davon, dass die geforderte Anzahl der anschließbaren Sensoren derzeit nicht erreicht wird, ist der Messaufbau einsatzbereit und die verbleibenden Probleme können mit überschaubarem Aufwand gelöst werden.
\newpage
\bibliography{Quellen/Quellen.bib}
\bibliographystyle {ieeetr}
\newpage
\section*{Anhangsverzeichnis}

\noindent \textbf{AMS\_Firmware}\\
\indent AS7261\_complete.bin\\
\indent AS7265\_complete\_moonlight\_v1\smallskip

\noindent \textbf{Dokumente}\\
\indent AS726x\_Design\_Considerations.pdf\\
\indent AS7261\_Datasheet.pdf\\
\indent AS7265x\_Datasheet.pdf\\
\indent lTC4316-I2C-Translator\_Datasheet.pdf\\
\indent Program\_AS72xx\_FlashCatUSB.pdf\smallskip

\noindent \textbf{Hardware}\\
\indent Sensorboard\_Layout.pdf\\
\indent Sensorboard.brd\\
\indent Sensorboard.sch\\
\indent Status-Adapterboard\_Layout.pdf\\
\indent Status-Adapterboard.brd\\
\indent Status-Adapterboard.sch\smallskip

\noindent \textbf{Programmcode}\\
\indent default\_values.h\\
\indent Makefile\\
\indent lib\\
\indent \indent influxDB\_http\_Libary\\
\indent \indent \indent influxdb.c\\
\indent \indent \indent influxdb.h\\
\indent \indent wiringPi\_AS726X\_Libary\\
\indent \indent \indent AS726X.c\\
\indent \indent \indent AS726X.h\\
\indent src\\
\indent \indent main.c\\
\indent \indent measurement.c\\
\indent \indent measurement.h\\
\indent \indent welcome.c\\
\indent \indent welcome.h
\end{document}