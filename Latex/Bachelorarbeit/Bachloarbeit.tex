\documentclass[a4paper,oneside,12pt,titlepage]{scrartcl}   %Grundeinstellungen

\usepackage[ngerman]{babel}
%\usepackage[T1]{fontenc} % umlaute
\usepackage[utf8]{inputenc}
\usepackage{amsfonts}
\usepackage{amsmath}
\usepackage{amssymb}
\usepackage{amsthm}
\usepackage{subfiles} 
\usepackage{caption}
\usepackage{wrapfig}
\usepackage[amssymb]{SIunits}
\usepackage{cite}
\usepackage[usenames,dvipsnames]{xcolor}
\usepackage{listings}
\usepackage{textcomp}

\definecolor{comment}{RGB}{0,128,0} % dark green
\definecolor{string}{RGB}{255,0,0}  % red
\definecolor{keyword}{RGB}{0,0,255} % blue

\lstdefinestyle{c}{
	commentstyle=\color{comment},
	stringstyle=\color{string},
	keywordstyle=\color{keyword},
	basicstyle=\footnotesize\ttfamily,
	numbers=left,
	numberstyle=\tiny,
	numbersep=5pt,
	frame=lines,
	breaklines=true,
	prebreak=\raisebox{0ex}[0ex][0ex]{\ensuremath{\hookleftarrow}},
	showstringspaces=false,
	upquote=true,
	tabsize=2,
}
 \lstset{language=c} 


\renewcommand{\familydefault}{\sfdefault} 
\usepackage{graphicx}
\usepackage{subcaption}                       
\usepackage{float}
\usepackage[section]{placeins}
\renewcommand{\topfraction}{0.85}
\renewcommand{\textfraction}{0.1}
\renewcommand{\floatpagefraction}{0.75}

\usepackage{cite}
\usepackage[usenames,dvipsnames]{xcolor}


\usepackage{listings}

\lstset{%
	 basicstyle=\scriptsize\ttfamily,
   keywordstyle=\bfseries\ttfamily\color{NavyBlue},
   stringstyle=\color{Rhodamine}\ttfamily,
   commentstyle=\color{Green}\ttfamily,
   emph={square}, 
   emphstyle=\color{blue}\texttt,
   emph={[2]root,base},
   emphstyle={[2]\color{yac}\texttt},
   language=Python,%
   tabsize=2,%
   basicstyle=\footnotesize\ttfamily,%
   numbers=left,%
   numberfirstline,%
   breaklines=true,%
   breakatwhitespace=true,%
   linewidth=\textwidth,%
   xleftmargin=0.075\textwidth,%
   frame=tlrb,%
   captionpos=b%
}


%Header Definitionen
\usepackage{fancyhdr}
\renewcommand{\headrulewidth}{0.5pt}
\renewcommand{\footrulewidth}{0.5pt}
%Abstand zwischen Absätzen, Zeilenabstände
\voffset26pt 
\parskip6pt
%\parindent1cm  %Rückt erste Zeile eines neuen Absatzes ein
\usepackage{setspace}
\onehalfspacing

\begin{document}
\pagenumbering{arabic}

\titlehead
{
	\begin{minipage}{0.6\textwidth}
	Technische Universität Berlin\\
	Fakulät IV\\
	Institut für Energie- und Automatisierungstechnik Fachgebiet Lichttechnik\\
	\end{minipage}
	\hfill
	\begin{minipage}{0.3\textwidth}\raggedright
	\includegraphics[scale=0.06]{tu-logo}\\
	\end{minipage}
	
	}

\title {Entwicklung und Realisierung einer Messeinrichtung mit den Sensoren AS7261 und AS72651 von ams}
\subtitle{Bachlorarbeit}
\author{Vorgelegt von: Lennard Bödiger\\
Matrikelnr.: 363470\\
Studiengang: Technische Informatik}

\maketitle  %Erstellt das Titelblatt wie oben definiert

%\tableofcontents
\section{Einleitung}
\newpage
\subfile{sections/Einleitung}
\section{Technische Grundlagen}
\subfile{sections/Technische_Grundlagen}
\section{Hardware Komponenten}
\subfile{sections/Hardware_Komponenten}
\newpage
\section{Platine}
\subfile{sections/Platine}
\section{Datenbank \& Webinterface}
\subfile{sections/Datenbank_und_Webinterface}
\section{C Code}
\subfile{sections/Software}

\section{Benutzerhandbuch}
\subfile{sections/Benutzerhandbuch}
\section{Messungen}
\subfile{sections/Messungen}
\section{Zusammenfassung}

\bibliography{Quellen}
%\cite{MT_Kap5}
\bibliographystyle {plain}


\end{document}