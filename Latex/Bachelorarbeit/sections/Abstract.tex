The goal of this work is to develop a spectral sensor platform using the AS7261 and AS72651, which can be used modularly for different research purposes, but especially for daylight measurement. Since it is not possible to measure in several directions in the sky at the same time with only one of the AS7261 (or AS72651) sensors without mechanical components, it must be ensured that a large number of sensors can be combined into a sensor array in order to be able to measure directional resolution. 
The resulting sensor platform uses a Raspberry Pi 4 as central control unit, which communicates with external sensor boards via I2C.
The sensor boards are designed to be connected in series to allow a variety of measurement configurations.
The database and the web interface for displaying and querying the data runs locally on the Rasperrypi.
The thesis describes the developed of all hardware and software components.


Ziel dieser Arbeit ist es, mithilfe des AS7261 und AS72651 eine Spektral-Sensor Plattform zu entwickeln, die modular für unterschiedliche Forschungszwecke, aber vor allem zu Tageslichtmessung, genutzt werden kann.Entscheidend dabei ist, dass das System portabel und kosteneffizient umgesetzt wird, um vielfache und flexible messaufbauten zu ermöglichen. Da es ohne mechanische Komponenten nicht möglich ist, mit nur einem der AS7261 (oder auch AS72651) Sensoren in mehrere Richtungen am Himmel gleichzeitig zu messen, muss es gewährleistet sein, eine Vielzahl der Sensoren zu einem Sensor Array zusammenzuschließen, um so Richtungsauflösend messen zu können. 

Die resultierende Sensorplattform nutzt einen Raspberry Pi 4 als zentrale Steuereinheit, welcher über I2C mit externen Sensnorboards kommuniziert.
Die Sensnoroards sind darauf ausgelegt, in reihe geschaltet zu werden, um so eine Vielzahl von Messkonfigrationen zu ermöglichen.

Die Datenbank und das Webinterface zum darstellen und abfragen der daten laufen lokal auf dem Rasperrypi.

In der Arbeit sind die entwickelten hard und Softwarekomponente beschrieben.
