Die Abbildungen \ref{fig:Grafana_Gain0}-\ref{fig:Grafana_Gain3} zeigen die gleiche Messung eines wolkenfreien Dezembertags am 12.12.2020 in Berlin mit unterschiedlichen Verstärkungsfaktoren der Integrationswert ist auf 255 gesetzt.\\
Es ist zu erkennen, dass die Auflösung der Y-Achse mit zunehmender Verstärkung (Gain) steigt. Aus Abbildung \ref{fig:Grafana_Gain3} ist jedoch ersichtlich, dass der maximale Messwert von 65.000 mit einem hohen Verstärkungsfaktor erreicht werden kann und die Daten so ihre Aussagekraft verlieren (clipping).
Abbildung \ref{fig:Grafana_AutoGain} zeigt die gleiche Messung im AutoGain-Modus, 
in diesem Plot werden immer die Messwerte mit dem größtmöglich Verstärkungsfaktoren, bei dem es nicht zu clipping kommt, zusammengefasst.
Die Ausgabe-Daten der Messung sind nicht kalibriert.
Außerdem haben sie keine Einheit, bei festem Gain liegen sie zwischen 0 und 65000.
Im Autogain-Modus liegen sie zwischen 0 und 4160000.
Um die Daten nutzbar zu machen, müssen sie mithilfe eines anderen Messgerätes auf allgemein verständlichen Wert kalibriert werden.

\begin{figure}[H]
  \begin{subfigure}[b]{0.5\textwidth}
    \includegraphics[width=\textwidth]{img/Grafana-Gain0}
    \caption{Gain 0 (1x)}
	\label{fig:Grafana_Gain0}
  \end{subfigure}
  %
  \begin{subfigure}[b]{0.5\textwidth}
    \includegraphics[width=\textwidth]{img/Grafana-Gain1}
    \caption{Gain 1 (3,7x)}
      \label{fig:Grafana_Gain1}
  \end{subfigure}
\end{figure}

\begin{figure}[H]
  \begin{subfigure}[b]{0.5\textwidth}
    \includegraphics[width=\textwidth]{img/Grafana-Gain2}
   \caption{Gain 2 (16x)}
	\label{fig:Grafana_Gain2}
  \end{subfigure}
  %
  \begin{subfigure}[b]{0.5\textwidth}
    \includegraphics[width=\textwidth]{img/Grafana-Gain3}
	\caption{Gain 3 (64x)}
	\label{fig:Grafana_Gain3}
  \end{subfigure}
\end{figure}


\begin{figure}[H]
\centering
\includegraphics[width=0.6\textwidth]{img/Grafana-AutoGain}
\caption{AutoGain}
\label{fig:Grafana_AutoGain}
\end{figure}
